\section{Datasets}
\label{sec:datasets}
There are many databases, which provide a lot of datasets. One of those is \textit{Kaggle}. \textit{Kaggle} does not only provide datasets, they also organize competitions in analizing datasets or improving already invented algorithms. We found a lot of data about to \textit{New York Stock Exchange}. Unfortunately we did not find one single dataset which satisfied all our requirements. There was no data about how often a search term was requested at \textit{Google}. So we had to collect the data ourselves from \textit{Google Trends}. In the following chapters we describe our datasets and why we are using them.

\begin{itemize}
	\item{first of all the necessary data needs to be collected: \\
		\begin{itemize}
			\item google trends data for training
			\item stock data for validation
		\end{itemize}
	}
	\item stock data was easy to collect
	\item google trends data, however, was a quite hard task
\end{itemize}

\subsection{Stock data}
\label{subsec:stockdata}
nyse


\subsection{Google Trends data}
\label{subsec:gtdata}
google trends data

\subsubsection{Hacking Google Trends}
\label{subsub:hackinggt}
\begin{itemize}
	\item google trends has no api unlike many other google services
	\item therefore, an analysis of the request url was done to send prepared requests for data collection
	\item some additional problems like max. number of requests, request blockade etc.
\end{itemize}


\subsection{Merged dataset}
\label{subsec:merged}
merging...