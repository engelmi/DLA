\section{Conclusion}
\label{sec:ausblick}
In this paper 
+ results of the project to predict rising or falling stocks using deep learning methods are presented
+ the basic framework for
	- retrieving the google trends data
	- preprocess the data
	- learning models
	implemented
+ simple learning model, which uses a lstm, achieved an accuracy of 53\% , just above simple guessing 
+ after ca. 150 epochs with k fold cross validation (k=10) and dropout the accuracy reached its max - longer training was contra productive 

+ simplified approach therefore does not encourage further research of predicting stocks based on google trend data
+ however as only the lstm, and only a single cell, was used, other models and model combinations could yield better results
+ as the architecture for this project was implemented flexible for future extensions, it could serve as a starting point / core for this future research. for example, another model could be derived from the abstract class and used instead

simplified approach reached xx\% of accuracy, therefore it would be worthwhile to enhance the current model by predicting ranges of values, e.g. 50 up or down etc.
as described in \ref{sec:architecture}, only minor changes would only be necessary
\begin{itemize}
	\item summarize the results (implementation? evaluation results?) 
	\item does it work? why? why not? 
	\item how could the current solution be improved? 
	\item possible further research? 
	\item how fast must it be. Finance is fast changing market
\end{itemize}