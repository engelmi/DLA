\section{Conclusion}
\label{sec:ausblick}
In this work we showed how to build a basic framework for predicting the course of a stock by analyzing Google trends data using deep learning methods. The framework consists of two major parts. The first is about retrieving, preprocessing and reshaping data. Collecting the data from the Google trends service proved to be quite difficult. The second part is about the models used for stock prediction. We introduced in this paper two simple learning models based on a long short term memory.\\
The simplification to a binary classification task reduced the complexity of the prediction problem. Thus a first investigation of the hypothesis, which is stated in the introduction \ref{sec:introduction}, was easier. The hypothesis assumes a correlation between Google trends search terms and stock courses. The \textit{SimpleLearningModel}, which uses all the data from Google trends, achieved an accuracy only slightly above 50\% which is as bad as our second model using no data from the Google trends service. Even by simple guessing the rising or falling of the stocks a similar result can be achieved. Newcomer to the field of stock exchange and financial experts may be even better than our models as they can use well defined and proven financial models for stock prediction. Unfortunately, our simple approach of using Google trends data for predicting the stock courses does not encourage further research. Another approach, however, with more structured retrieval and shaping the Google trends data may still be worthwhile. The data used in this project from Google trends often had no correlation to the company or stock we tried to predict. The usage of similar search terms provided by Google were not constructive. By using more precise related search terms and topics, in combination with structural changes, better results could be achieved. Also, as only the long short term memory with a single cell was investigated, other models and model combinations could yield better results. There exist other gated recurrent neural networks which also do not have the problem of exploding or vanishing gradients. A completely different approach would be the usage of convolutional networks. The architecture of the framework is flexible enough to easily try these other models, whereas it can be used as a starting point for future research.
\\
Due to the fact that no API is available for this service, the dataset used in this project could have been too small. A larger dataset of Google trends could, therefore, also improve the accuracy of the predictions. If Google provides an API in the future for the Google trends data, collecting a more comprehensive dataset gets easier and our models could perform better. The number of similar search terms used for prediction could also be enlarged from 14 up to 20 terms or more. It is also possible to increase or decrease the sequence length of 30 days.
\\
As already stated, the achieved prediction results do not encourage a more complex approach to predict price ranges or even the exact price. Also, as financial models for price prediction are far more sophisticated, the use of a deep learning model for stock prediction is still not advisable. Overall it can be said that Google trends data does not improve the prediction of stocks. But our dataset was small and often the data did not have any relation to the company we tried to predict the stock for.
