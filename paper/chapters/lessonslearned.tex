\section{Lessons Learned}
\label{sec:lessonslearned}

\subsection{Daniel Sikeler}
\label{sec:llDS}
The idea of predicting the trend of a stock price is very interesting because I did some investment in stocks and could use it myself. Unfortunately it was difficult to get the data we needed. It took us a long time to get the scripts running for data collection and then \textit{google trends} we ran into the problem of limit number of requests. So in the end we had to collect data every day. Additionally, we often got data from \textit{google trends} with a lot of zeros and so the data was unfeasible for us. We had to review the data and remove the keywords with useless data.\\
Deep learning is complex and needs a lot of background information to understand what is going on during training a model and how to improve the performance and accuracy of the model. At the beginning it was a lot of "try and error" because tensorflow further increased the complexity. The tensorboard documentation is so rich on information that you can spend a lot of time there if you do not know exactly what you are looking for. But tensorboard also has some advantages as it hides a lot of the internals of an LSTM cell.\\
If you are ready building the model there is still a lot to do. The training process is very time consuming. You need a lot of data, and in my opinion we did not have enough, to achieve good results with the trained model. We had to decide if we spent more time on collecting further data or try to improve our model. We went for understanding our model and try to improve it. We trained the model with different learning rates, numbers of hidden layers in one LTSM cell. We also tried some other activation and loss function which did not really improve the model.\\
Evaluating the model is not simple. The accuracy and loss can be easily logged but if there is a lot of variation it is hard to say how to improve the training. For me it is useful to understand what is going on inside the model, but it is nearly impossible to say what the weights are representing. And so we can not really say for what the model is looking for.\\
Summarizing my experience I can say that it was a interesting task but I should have spent more time on tensorboard and collecting data which was not possible because of other classes. I like to keep myself busy on the topic of deep learning and machine learning in general. 

\subsection{Michael Engel}
\label{sec:llME}
Predicting the development of stock prices is an interesting research topic as stocks are still considered to be unpredictable. The fact that stock prices are greatly influenced by the actions of the traders led to the basic idea of using user generated data and feed it to a deep learning model. Although the binary classification was solved just about 50\% of the time, this result could have been worse. The approach of using user generated data for stock prediction could therefore yield quite promising results if aspects like the quality and quantity of data are enhanced. 
\\
Unfortunately, there were a few unexpected obstacles that made the complex task of developing a deep learning model for stock prediction even harder. The non-existing API for the Google trend service entailed an time consuming analysis of the http request send by this website to retrieve the needed information. Also, the used machine learning library TensorFlow is comprehensive and initially overwhelming for a beginner in the field of deep learning. The correct shape of data across the multiple operations and their correlations caused some trouble. However, after this initial training many features of TensorFlow, like the DropoutWrapper, could be used quite easily. The use of a framework like Keras would have accelerated the whole development process nevertheless. The efficient use of data as well as the interpretation of any results was unexpectedly difficult concerning the improvement of the prediction results. 
\\
Overall, deep learning is a fascinating research topic. It's versatility and applicability in a multitude of domains contributes much to this fascination. Additionally, it is delightful to read scientific papers which were written in the last few years instead of the last century. During the task of using deep learning to predict the allegedly unpredictable stock prices the essentials of deep learning were learned and the desire to do more projects in this domain has been awakened. 